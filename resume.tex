\documentclass{resume} % Use the custom resume.cls style

\usepackage[left=0.4 in,top=0.4in,right=0.4 in,bottom=0.4in]{geometry} % Document margins
\newcommand{\tab}[1]{\hspace{.2667\textwidth}\rlap{#1}} 
\newcommand{\itab}[1]{\hspace{0em}\rlap{#1}}
\name{Sanjay Mohan Kumar} % Your name
% You can merge both of these into a single line, if you do not have a website.
\address{\href{mailto:smohanku@gmu.edu}{smohanku@gmu.edu} \\ +1 (571)307-0134} 
\address{\href{https://linkedin.com/in/smohanku}{linkedin.com/in/smohanku} \\ \href{https://github.com/San68bot}{github.com/San68bot}}  %

\begin{document}

%----------------------------------------------------------------------------------------
%	EDUCATION SECTION
%----------------------------------------------------------------------------------------
\vspace{-1.5em}
\begin{rSection}{Education}

{\bf George Mason University} \hfill {Fairfax, VA}\\
Pursuing Bachelor of Science in Computer Science \hfill{\textit{Exp. Graduation: May 2026}}\\
Relevant Coursework: Data Structures, Low-Level \& Systems Programming, OOP\\
Awards/Certifications: Dean's List 2023, Microsoft Office Specialist Master Certification

\end{rSection}

%----------------------------------------------------------------------------------------
% TECHINICAL STRENGTHS	
%----------------------------------------------------------------------------------------
\begin{rSection}{SKILLS}

\begin{tabular}{ @{} >{\bfseries}l @{\hspace{6ex}} l }
Programming & Java, Kotlin, Python, C/C++, Linux/Unix, LaTeX, R \& R Studio\\
Technology & Git, vSLAM, OpenCV, TensorFlow, Valgrind \& GDB\\
CAD & Onshape, Autodesk Fusion360, 3D Printing, Rendering, Surface Modeling\\
Hardware & Microcontrollers, Embedded Systems, PCB Design, IoT\\
Robotics & FSM, Control Systems, Motion Planning, Kinematics, Sensors Programming\\
\end{tabular}
\end{rSection}

%----------------------------------------------------------------------------------------
%	WORK EXPERIENCE SECTION
%----------------------------------------------------------------------------------------

\begin{rSection}{EXPERIENCE \& LEADERSHIP}

\textbf{FIRST Alumni Association, Vice President} \hfill Oct 2022 - Present
 \begin{itemize}
    \itemsep -3pt {} 
     \item Led STEM-focused club as Vice President, organizing FIRST robotics events and community outreach to drive interest in robotics and STEM related fields.
     \item Helped lead "Spring into STEM" community event at GMU which promoted STEM to students K-12th in the DMV area, with roughly 300-500 participants.
    \item Through various marketing and promotion strategies, grew the alumni association by 20-35 members over the course of two semesters.
 \end{itemize}

\textbf{GMU Undergraduate Teaching Assistant} \hfill Aug 2023 - Present
 \begin{itemize}
    \itemsep -3pt {}
     \item Supported roughly 200+ students each week in topics of Low-Level C Programming and Unix fundamentals, Supported GTAs and Instructors during weekly labs and Lecture classes, Code reviewed and provided intuitive constructive feedback for Labs and Programming Assignments.
 \end{itemize}

\end{rSection} 

%----------------------------------------------------------------------------------------
%	PROJECTS SECTION
%----------------------------------------------------------------------------------------

\begin{rSection}{PROJECTS}
\vspace{-1.25em}
\item{\underline{\textbf{Smart Signal}}}: \textit{C/C++ $\vert$ Arduino} \hfill {Aug 2017 - Dec 2017} \vspace{0.25em} \\
{I created a dynamic traffic intersection simulation, using ultrasonic sensors to prioritize the busiest lane for the green light, reducing traffic congestion and wait times. This project gained recognition for seamlessly integrating hardware and software solutions, highlighting my proficiency in addressing practical problems through innovative technology.}

\item{\underline{\textbf{Finite State Machine Builder}}}: \textit{Finite State Machine $\vert$ Software Library} \hfill {Jan 2022 - Apr 2023} \vspace{0.25em} \\
{An extremely robust software library designed to make the creation and analysis of Finite State Machines as easy as possible. Its potential to be embedded into any system is made possible by a "Plug-and-Play" style interface.}

\item{\underline{\textbf{Adaptive Sensor Localization}}}: \textit{Real-World Localization $\vert$ Sensor Fusion} \hfill {May 2022 - Jun 2023} \vspace{0.25em}\\
{Software library that uses distances to nearby fixed objects to derive a robot's relative real-time position (x, y, $\theta$). Additionally, it can communicate with sensors to retrieve readings at optimal times as needed.}

\item{\underline{\textbf{AlphaLib Robot Software Library}}}: \textit{Trajectory Generation \& Following $\vert$ OpenCV} \hfill {Jun 2021 - Jun 2023} \vspace{0.25em}\\
{An all-in-one software package that makes programming any FTC robot easier and highly efficient. Several control theory concepts are skillfully implemented, such as Trajectory Generation \& Following, Open(FeedForward) \& Closed(PID) Loop Controllers, and Object Classification.}
\end{rSection} 

\end{document}