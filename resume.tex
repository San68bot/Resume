\documentclass{resume} % Use the custom resume.cls style

\usepackage[left = 0.4in, top = 0.4in, right = 0.4in, bottom = 0.4in]{geometry} % Document margins
\newcommand{\tab}[1]{\hspace{.2667\textwidth}\rlap{#1}} 
\newcommand{\itab}[1]{\hspace{0em}\rlap{#1}}
\name{Sanjay Mohan Kumar}

\address{
    571-307-0134 \\ 
    \href{mailto:smohanku@gmu.edu}{ smohanku@gmu.edu }
}

\address{
    \href{www.linkedin.com/in/smohanku}{ linkedin.com/in/smohanku } \\ 
    \href{https://github.com/San68bot}{ github.com/San68bot }
}

\begin{document}

    %----------------------------------------------------------------------------------------
    %	EDUCATION SECTION
    %----------------------------------------------------------------------------------------

    \begin{rSection}{Education}
        {\bf George Mason University} \hfill {Expected Graduation: May, 2026}\\
        Pursuing Bachelors degree in Computer Science \hfill \textit{Fairfax, VA}
        \vspace{-0.5em}
        \begin{itemize}
            \itemsep -3pt {} 
            \item {\bf Relevant Coursework:} {Object-Oriented Programming, Data Structures}
        \end{itemize}
        
        {\bf Broad Run High School} \hfill {Aug 2018 - June, 2022}
        \vspace{-0.5em}
        \begin{itemize}
            \itemsep -3pt {} 
            \item {\bf Advanced Placement:} {World History, Computer Science A, Statistics} \hfill \textit{Ashburn, VA}
            \item {\bf Honors:} {Research Chemistry, Physics, Project Lead the Way(PLTW)}
            \item {\bf Achievments:} {Workplace Readiness Certification, Microsoft Office Specialist}
        \end{itemize}
    \end{rSection}

    %----------------------------------------------------------------------------------------
    %   SKILLS SECTION
    %----------------------------------------------------------------------------------------
    \begin{rSection}{SKILLS}
        \begin{tabular}{ @{} >{\bfseries}l @{\hspace{6ex}} l @{\vspace{0.5ex}} l }
            Programming & Java, Kotlin, Python, C++, Latex, Git, Unix Shell, OpenCV\\
            Software & IntelliJ, Visual Studio, MATLab, Microsoft Office, Adobe Software\\
            Hardware & Arduino, Raspberry Pi, 3D Printing\\
            CAD & Onshape, Autodesk Fusion 360\\
            Robotics & State Machines, Control Theory, Trajectory Generation \& Following, Open \& Closed Loop Control\\
            Soft Skills & Leadership, Problem Solving, Critical Thinking, Teamwork\\
        \end{tabular}\\
    \end{rSection}
    
    %----------------------------------------------------------------------------------------
    %   PROJECTS SECTION
    %----------------------------------------------------------------------------------------

    \begin{rSection}{PROJECTS}
        \vspace{-1.25em}
        \item \textbf{Smart Signal}: \textit{C++ $\vert$ Arduino} \hfill {September, 2017}\\
        {Built a tool to search for Hiring Managers and Recruiters by using ReactJS, NodeJS, Firebase and boolean queries. Over 25000 people have used it so far, with 5000+ queries being saved and shared, and search results even better than LinkedIn! \href{https://hiring-search.careerflow.ai/}{(Try it here)}}

        \item \textbf{Projectile Motion Simulator}: \textit{Python $\vert$ Physics} \hfill {September, 2020}\\
        {Build a project that does something and had quantified success using A, B, and C. This project's description spans two lines and also won an award.}

        \item \textbf{State Machine Builder}: \textit{Kotlin $\vert$ State Machines} \hfill {February, 2022}\\
        {Build a project that does something and had quantified success using A, B, and C. This project's description spans two lines and also won an award.}

        \item \textbf{Sensor Localization}: \textit{Kotlin $\vert$ Control Theory $\vert$ Real-Time Position Estimation} \hfill {July, 2022}\\
        {Build a project that does something and had quantified success using A, B, and C. This project's description spans two lines and also won an award.}

        \item \textbf{AlphaLib}: \textit{Kotlin $\vert$ Control Theory $\vert$ Trajectory Generation \& Following $\vert$ OpenCV} \hfill {August, 2022}\\
        {Build a project that does something and had quantified success using A, B, and C. This project's description spans two lines and also won an award.}
    \end{rSection} 

\end{document}